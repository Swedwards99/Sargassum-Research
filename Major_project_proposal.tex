\documentclass[12pt]{extarticle}
\usepackage[utf8]{inputenc}
\usepackage{cite}
\usepackage{graphicx}
\usepackage[english]{babel}
\usepackage{amsmath}
\usepackage{csquotes}
\usepackage{placeins}
\usepackage{textcomp}
\newcommand\x{\textit{S. muticum}}
%%%Title%%%
\title{\textit{Sargassum muticum} Research Proposal}
\author{Spencer Edwards \\
BIOL 456 - Algae Lab \\
\textsc{Western Washington University}}
\date{May 18, 2020}
%%%%%%%%%%%%%%%%%%%%%%%%%%%%%%%%%%%%%%%%%%%%%%%%%%%%%%%%%%%%%%%%%%%%%%%%%%%%%%%%%%%
\begin{document}

\maketitle
\section{Title}
An Investigation into Potential Phenotypic Plasticity of \x\

\section{Introduction}
\paragraph{Research Question}
The ecology of invasive species is of growing concern in an era of large-scale changes to the planet. As the climate warms due to human activity, Non-Indigenous Species (NIS) spread northward, colonizing new habitat and interacting with indigenous species. \cite{de2006northward}. Further transfer of NIS can occur from direct human transfer, either intentional or accidental. Recreational boating, for instance, is one of the largest unregulated vectors of invasive marine species \cite{Murray}. There are a few key questions when analyzing the ecological causes and consequences of such invasions. What causes an invasive species to become successful? Do NIS have greater levels of genetic diversity than indigenous species, or is some other mechanism at play? How do NIS interact with indigenous species to effect changes to the ecosystem as a whole?
\section{Literature Review}
\paragraph{Background}
Current research on \textit{Sargassum muticum} (Yendo) Fensholt and its global spread has been heavily region-specific \cite{Britton-Simmons2004, Staehr2000, Sanchez2005}. The specificity of these reports means little can be extrapolated  concerning the global impact of the \textit{S. muticum} invasion. Furthermore,  research regarding the impact of the invasion on macroalgal and inter-tidal assemblages  fails to explain how and why \textit{S. muticum} is so successful. Research by \textit{Le Cam et. al.} found extremely low genetic diversity in worldwide populations of \textit{S. muticum}, despite its success. \cite{LeCam2019}. This suggests \textit{S. muticum} has an extremely high tolerance for different environmental conditions, and that  phenotypic plasticity may be occurring in the meta-population, allowing \x\ to adapt as it spreads to different regions. The \textit{`Plastic Increase in Competitive Ability Hypothesis’}, as outlined by Frome and Dybdal, and expanded upon by Schwartz et. al., suggests that when invasive species extend their ranges to areas where they face little competition, resources usually allocated to defense can instead be used for growth and reproduction \cite{EmblidgeFromme2006}. When they tested \textit{S. muticum}'s anti-fouling defense in its native and introduced ranges, the authors found: 
% Block Quote %%%%%%%%%%%%%%%%%%%%%%%%%%%%%%
\begin{displayquote}
``Overall, extracts of \textit{S. muticum} from the invasive range were less active compared to those of the native range suggesting an adaptation to lower fouling pressure and competition in the new range resulting in a shift of resource allocation from costly chemical defence to reproduction and growth" \cite{Schwartz2017}.
\end{displayquote}
 Research on the green algae \textit{Desmodesmus} and
\textit{Scenedesmus} revealed that the algae responded to chemical cues to herbivorous zooplankton by forming eight-celled colonies during development \cite{Lurling2003}. Furthermore, \textit{Miner et. al.}, in their paper \textit{Ecological Consequences of Phenotypic Plasticity}, described \textit{the Paradox of Enrichment}, which states that as nutrients are added to an environment, large fluctuations could make populations go extinct. One answer to this paradox, they argue, is plasticity. They state: 


% Blockquote%%%%%%%%%%%%%%%%%%%%%%%%%%%%%%%%%%%
\begin{displayquote} ``In an empirical test of the predictions of theoretical models, Verschoor et al. \cite{verschoor2004inducible} found similar results in experiments with aquatic organisms. They demonstrated that population fluctuations were much less in a system with an alga, herbivorous zooplankton, and carnivorous zooplankton when the species of alga had an inducible defense than when it did not" \cite{Miner2005}. \end{displayquote}
This could potentially be a driving factor in how \textit{S. muticum} benefits local assemblages from further invasion, by increasing the overall diversity of local assemblages, referred to as the \textit{Diversity Resistance Hypothesis} \cite{Incera2009}. Indeed, \textit{Sanchez et. al.} argued that \textit{S. muticum} may provide benefits to oppurtunistic local species due to its large epiphitic assemblage \cite{Sanchez2005}. In their meta analysis of current research on \textit{S. muticum}, \textit{Engelen et. al} explained how \x\ displays high levels of plasticity of growth, owing to a complex modular growth form. They also pointed out that this is a mechanism by which \x\ prevents invasions by other species \cite{Engelen2015}.


Synthesizing all this information, I propose a new hypothesis: the ``\textit{Benevolent Warlord Hypothesis}". \textit{S. muticum}, aided by a highly plastic development, invades local assemblages, but also protects them from further invasion. The goal of this study will be to determine what, if any, plasticity \textit{S. muticum} displays in response to environmental cues. Once the degree of plasticity is quantified, further research could be conducted to determine the impact of this plasticity on the community ecology of \textit{S. muticum} and its competitors. Specifically, I hypothesize that \x\ displays high levels of phenotypic plasticity in its morphology in response to variation in nutrient levels and wave stress.


%%%%%%%%%%%%%%%%%%%%%%%%%%%%%%%%%%%%%%%%%%%%%%
\paragraph{Rationale}
To date, investigation of the invasive brown alga \textit{Sargassum muticum} has focused on its effect on local assemblages. However, to quantify the reason for its success, a new approach is needed. Using a Random Regression Mixed Model (RRMM) will allow us to test multiple fixed effects (for example, the effect of water temperature or salinity on the growth rate), as well as random effects (variation around different genotypes). \textit{Incera et. al.} found that \x\  grew best in constant, high nutrient conditions \cite{Incera2009}, so testing phenotypic change in response to growth could help further clarify those results, as well as suggest which areas it is likely to colonize next. Furthermore, research conducted off the coast of Ireland suggested that phenotypic plasticity was responsible for observed changes in development between individuals grown in high-energy vs low-energy environments (exposed vs sheltered shores) \cite{Baer2010}. In their article analyzing methods of quantifying plasticity, \textit{Rahaman et. al.} argued that along with genotypic variation, analyzing traits such as blade length and dry weight could be combined with other data, including genotypic data to provide further robustness \cite{10.3389/fpls.2015.00619}.
% I'm going to add more to this section
% Adding more information about the RRMM will help the reader.

%%%%%%%%%%%%%%%%%%%%%%%%%%%%%%%%%%%%%%%%%%%%%%%%%%%%%%
\section{Research Methodology}
\paragraph{Mathematical Background}
To test the idea that phenotypic plasticity is a key to the success of \textit{Sargassum muticum}, we will conduct a factorial growth experiment. We will run an experimental study and a literature review (meta-analysis). The experiment will involve morphological analysis as well genotypic analysis, using random regression mixed model framework to plot reaction norms by phenotype, as outlined by \textit{Arnold et. al.} \cite{Arnold2019}. A standard Random Regression Mixed Model is defined as 
\[y_\text{ij} = \alpha + \beta x_i + a_i + bx_i + e_\text{ij}\]
where the intercept \(\alpha\) and slope \(\beta\) are fixed components, which describe the population mean reaction norm, and a and b represent the random (individual level) intercepts and slopes, respectively \cite{Arnold2019}. RRMMs can also incorporate non-linear equations by adding terms to the standard equation \cite{Arnold2019}. The generalized equation for a polynomial RRMM is
\[y_\text{ij} = \alpha + \beta_1 x_i + \beta_2 x_i^2 + \dots + \beta_n x_i^\text{n}\]
We can use RRMMs to quanitfy the reaction norms of \x\ by environment and by genotype, allowing us to quantify plasticity for a large amount of variables.
\paragraph{Phenotypic Data}
In their review of developments in the quantitative analysis of phenotypes, \textit{Rahaman et. al.} argued that using an imaging system would allow for researchers to study developments in morphology such as biomass and shape \cite{10.3389/fpls.2015.00619}. Qualitative image analysis using \textit{ImageJ}, as well as performing dry weighing and microscopy, will be used to covert image data into raw mathematical data, which then can be fed into the model. For this expiriment, \x\ will be collected from Vendovi Island, WA, allowed to grow and reproduce, and gametes collected for the expriment. These gametes will be placed into the growth tanks at UW Friday Harbor Labs. The length of the thallus will be measured using calipers or rulers for each individual every day, and recorded. They will be measured from the bottom of the stem to the tip of the tallest portion of the thallus each time, to maintain accuracy. At the end of the experiment, specimens will be dried and weighed on a Precision Balances PL602E\textsuperscript{\textcopyright} by Mettler-Toledo, to the nearest hundredth of a gram.
\paragraph{Simulated Environmental Factors} By growing \textit{S. muticum} in the lab, it will be possible to control for the different environmental factors we will test. Simulating variation in nutrient availability will be done by replicating \textit{Incera et. al.}'s \cite{Incera2009} methods in the lab, rather than in the field. By growing \x\ in the lab, we can precisely control the amount of nutrients, as well as monitor growth on a daily basis. For the factorial expiriment, we will be modifying and replicating \textit{Incera et. al.}'s methods for the nutrient factor. They used 3 kg (2 mesh bags containing 1.5 kg each for High Intensity) and 1 kg
(2 mesh bags containing 0.5 kg each for Low Intensity) of slow,
controlled-release fertilizer (Multicote \textsuperscript{\textregistered} 4). The mesh bags will placed on 2 sides of the growth tanks Procedural controls used similar mesh bags filled
with a plastic bag containing sand. We will use Multicote\textsuperscript{\textregistered}
4 containing a ratio of 15\%  N (8\% NH4+, 7\% NH3), 7\%
P (PO2), 15\% K (K2O) and 2\% MgO for a control, as they did. To test the effect of wave energy on growth, tanks will be randomly assigned a fan to generate turbulance, or no fan, simulating a calm tide pool. 
\paragraph{Genotypic Data}
By replicating \textit{Le Cam et. al.}'s methods--using a ddRAD to analyze SNPs-- we should be able to replicate their results, and use the observed genotypes to run our analysis. As was done in \textit{Le cam et. al.}, we will use 454 GS‐FLX titanium technology to obtain microstalite loci, and 120 $\mu$L of buffer solution from the Nucleospin 96 Plant kit and PCR to sequence the genotypes.
%%%%%%%%%%%%%%%%%%%%%%%%%%%%%%%%%%%%%%%%%%%%%%%%%%%%%%%%%%%%%%%%%%%%%%
\section{Plan of Work}


\FloatBarrier
\bibliographystyle{plain}
\bibliography{M335}

\end{document}
